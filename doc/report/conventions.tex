\section{Conventions}

Benchmarks must be written in a common notation, called \co{Roops},
which can be characterized as an extended common subset of Java and C\#.

We provide several examples for \co{Roops} benchmarks together with translators to both Java and C\#, 
which can be easily adapted.

\subsection{Benchmark Files}

Several benchmarks are collected in a benchmark class that is stored in a file.
Each file must follow the following format.

\begin{itemize}
\item {\tt //\$} must be given on separate lines.
\item {\tt //\$category NAME} defines a package/namespace NAME. NAME should be of the form \co{aaaa.bb.ccc}. The file containing this tag should be in directory \co{aaaa/bb/ccc}.
{\it Motivation:} This will satisfy the Java package conventions and make it easy to generate Java files. 
\item {\tt //\$benchmarkclass} introduces a class containing benchmarks.
{\it Motivation:} Methods in Java and C\# must reside in classes.
\item {\tt //\$goals COUNT} must immediately precede a benchmark, defines that the benchmarks has goals $0, \ldots, COUNT-1$.
{\it Motivation:} We want to measure how many goals are covered by individual tools.
\item {\tt //\$benchmark} introduces a benchmark method.
{\it Motivation:} Code in Java and C\# is stored in methods.
\item {\tt //\$endcategory NAME} finishes a package/namespace NAME.
{\it Motivation:} Namespaces in C\# are scoped.
\end{itemize}

For simplicity of the tools, benchmarks have to be proper methods and cannot be constructors.
While many of the benchmarks in the repository only consists of single methods,
it is allowed for benchmark methods to call other methods,
and goals may be distributed over all (transitively) called methods.
However, the annotation stating how many goals there are in total has to be attached at the top-level benchmark method.

So, if you want to only consider a constructor as a benchmark, the idea is that you create a benchmark method for it with all the annotations, and then the benchmark invokes the constructor, which then contains the actual goals.

Within each benchmarks, goal statements are encoded as follows.
\begin{itemize}
\item {\tt \{/*\$goal NUMBER*/\}} must immediately precede a benchmark, defines that the benchmarks has goals $0, \ldots, COUNT-1$. 
This goal has an {\tt unknown} verdict.
\item {\tt \{/*\$goal NUMBER VERDICT*/\}} as before, but states a verdict, which must be one of the following:
\begin{itemize}
\item {\tt reachable}: There is at least one tool which could generate a reproducible test case that reaches this point.
\item {\tt unreachable}: This goal is unreachable, as far as we know.
\item {\tt unknown}: Neither of the above.
\end{itemize}
In the case of a competition, we suggest that the explicit verdict is stripped
from the program text before it is given to the tool.
\end{itemize}

For example, the following is an excerpt of the 
\verb|LinearWithoutOverflow| benchmark class,
which resides in \verb|roops/core/bv32/linear/noex/gods/|
in the \verb|src-benchmarks-roops| directory.

\begin{verbatim}
//$category roops.core.bv32.linear.noex.gods

// Authors: Nikolai Tillmann and Christoph Csallner
//$benchmarkclass
public class LinearWithoutOverflow
{
    #region Equals
    //$goals 2
    //$benchmark
    public void XPlusYEqualsZ(int x, int y, int z)
    {
        if (x + y == z)
            { /*$goal 0*/}
        else
            { /*$goal 1*/}
    }
		
		/* .. */
		
    #endregion
}
    
//$endcategory roops.core.bv32.linear.noex.gods		
}
\end{verbatim}

Exception handling must be done via the following syntactic entities,
that replace the usual \co{try}, \co{catch}, \co{finally}, \co{throw} statements.
\begin{itemize}
\item {\tt //\$try} must precede a protected block.
\item {\tt //\$catch} precedes an exception handler block following a protected block.
\item {\tt //\$finally} precedes a finally block following a protected block.
\item {\tt //\$throw} throws some exception.
\end{itemize}
Note that the exception handler cannot specify a particular exception type.
As a consequence, the catch handler may not observe the caught exception type,
or access the exception object in any other way.
In the corresponding Java and C\# code, the handler will catch the most general exception type.
See \refsection{MoreExamples} for examples.
\co{Roops} does not have Java's \co{throws} statement. 
{\it Motivation:} While the benchmark method itself probably won't throw exceptions,
it might be useful to have this statement available when adapting existing algorithms 
to the \co{Roops} notation.

The following syntactic entities allow an easy conversion between Java and C\#
in the presence of inheritance:
\begin{itemize}
\item {\tt extends/*\$:*/} and {\tt :/*\$extends*/}
\item {\tt implements/*\$:*/} and {\tt :/*\$implements*/} and {\tt ,/*\$implements*/}
\item {\tt @Override} and {\tt /*\$*/override}
\item {\tt @Virtual} and {\tt /*\$*/virtual}
\end{itemize}

\subsection{Discussion}

It is often desirable to impose bounds on inputs to benchmarks,
e.g.\ bounding the number of objects in the input graph, or the number of branches executed along individual code paths.
However, such bounds are highly tool specific, and are not yet included in the \co{Roops} notation.
