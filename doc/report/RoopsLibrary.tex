\label{RoopsLibrary}\section{Roops Library}

The benchmarks must not use any library class or methods which are not included in the benchmark.
Furthermore, Roops comes with its own standard library which can be implemented in Java and C\#:

\subsection{Arithmetic}
\begin{verbatim}
    public class RoopsInt32 {
        public static int MIN_VALUE = -2147483648;
        public static int MAX_VALUE = 2147483647;
        
        public static int Divide(int x, int y) {
            if (x == MIN_VALUE && y == -1) 
                return 0;
            else
                return x / y;
        }
        public static int Remainder(int x, int y) {
            if (x == MIN_VALUE && y == -1) 
                return 0;
            else
                return x % y;   
        }
    }
    public class RoopsInt64 {
        public static long MIN_VALUE = -9223372036854775808;
        public static long MAX_VALUE = 9223372036854775807;
        
        public static long Divide(long x, long y) {
            if (x == MIN_VALUE && y == -1) 
                return 0;
            else
                return x / y;
        }
        public static long Remainder(long x, long y) {
            if (x == MIN_VALUE && y == -1) 
                return 0;
            else
                return x % y;   
        }
    }
\end{verbatim}

Example:
\begin{verbatim}
    //$goals 2
    //$benchmark
    public void XRemainderYEqualsZ(int x, int y, int z) {
        if (y != 0 && RoopsInt32.Remainder(x, y) == z)
            { /*$goal 0 */}
        else
            { /*$goal 1*/}
    }
\end{verbatim}


\subsection{Array}
\begin{verbatim}
    public class RoopsArray {
        public static int getLength(int[] a) {
            // if a is null, throws exception;
            // otherwise, returns length of a
        }
    }
\end{verbatim}
