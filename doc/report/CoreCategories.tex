\section{Core Categories}

All benchmarks are categorized according to the following classifications.
Each benchmark may refer to multiple categories.
Benchmarks may call other methods which are included as source code,
but they may not refer to library classes or methods not mentioned below.

\subsection{Procedural Language}

At the core of all categories is a basic subset of C\# and Java
which defines methods and control-flow:
\begin{itemize}
\item Benchmark methods
\item Static helper methods
\item \co{if}, \co{while}, \co{for}, recursion
\item \co{switch} over integers
\end{itemize}

Supported datatypes are \co{int}, \co{long}, \co{boolean},
\co{Object}, and arrays over the above.
Allowable operations and required annotations are detailed in the categories below.

\subsection{Arithmetic}

\subsubsection{Linear Arithmetic for 32-bit Integers (without overflows)}

These benchmarks may only contain linear integer arithmetic (addition, multiplication with constants)
and integer relations. The reachability of branches is not affected by overflows of 32-bit integers. 
(A tool that can only reason about gods integers may not be able to determine the 
reachability of all branches.)
\begin{verbatim}
    //$goals 2
    //$benchmark
    public void XPlusYEqualsZ(int x, int y, int z) {
        if (x + y == z)
            { /*$goal 0 */}
        else
            { /*$goal 1*/}
    }
\end{verbatim}

\subsubsection{Linear Arithmetic for 32-bit Integers (with overflows)}

These benchmarks may only contain linear integer arithmetic (addition, multiplication with constants)
and integer relations. The reachability of branches may be affected by overflows of two's-complement 
32-bit integers. (A tool that can only reason about gods integers should not be able to 
determine the  reachability of all branches.)
\begin{verbatim}
    //$goals 2
    //$benchmark
    public void XIsSpecial(int x) {
        if (x < 0 && -x < 0) // 0x80000000 solves this
            { /*$goal 0 */}
        else
            { /*$goal 1*/}
    }
\end{verbatim}

\subsubsection{Non-Linear Arithmetic for 32-bit Integers}

These benchmarks may include multiplication of two non-constants,
bitwise operations (left shift, signed and unsigned right shift, or, and, xor, not).

\begin{verbatim}
    //$goals 2
    //$benchmark
    public void XTimesYOverflow(int x, int y) {
        if (x > 0 && y > 0 &&
            x * y < 0) 
            { /*$goal 0 */}
        else
            { /*$goal 1*/}
    }
\end{verbatim}

\subsubsection{64-bit Arithmetic}

Just as 32-bit arithmetic.

\subsection{Objects without exceptions}

Objects may be parameters of a benchmark. 
Interface and class types may be defined. 
All interfaces and classes must be public.
All fields must be \co{public}.
New objects may be created by the benchmark code,
fields can be written to or read from.
Casting is not allowed.
All dereferences of objects must be guarded by null-checks,
if necessary.
Generics are not allowed.
Interfaces may not extend other interfaces.
Static fields are not allowed.
All methods must be either \co{private} or \co{public}.
Custom constructors are not allowed; only the implicit nullary default constructor is allowed.
Any kind of annotation or modifier not specified in this section is not allowed,
e.g. \co{sealed}, \co{final}, \co{readonly}, \co{transient}, \co{synchronized}, \co{volatile}.
Explicit interface implementation and properties of C\# are not allowed.

Example 1:
\begin{verbatim}
    //$goals 2
    //$benchmark
    public void ObjectsNotEqual(Object x, Object y) {
        if (x != null && y != null &&
            x != y) 
            { /*$goal 0 */}
        else
            { /*$goal 1*/}
    }
\end{verbatim}

Example 1.5:
\begin{verbatim}
    //$goals 2
    //$benchmark
    public void ObjectsNotEqual(Object x) {
        Object y = new Object();
        if (x != null && y != null &&
            x != y) 
            { /*$goal 0 */}
        else
            { /*$goal 1*/}
    }
\end{verbatim}

Example 2:
\begin{verbatim}
    public class Point {
        public int X;
        public int Y;
    }
    //$goals 2
    //$benchmark
    public void PointsNotEqual(Point p, Point q) {
        if (p != null && q != null &&
            (p.X != q.X || p.Y != q.Y))
            { /*$goal 0 */}
        else
            { /*$goal 1*/}
    }
\end{verbatim}

Example 2.5:
\begin{verbatim}
    public class Point {
        public int X;
        public int Y;
    }
    //$goals 2
    //$benchmark
    public void PointsNotEqual(Point p, Point q) {
        if (p != null && q != null) {
            p.X = 42;
            if (q.X != 42)
                { /*$goal 0 */}
            else
                { /*$goal 1*/}
        }
    }
\end{verbatim}

Example 3:
\begin{verbatim}
    public interface IPoint {
        int GetX();
        int GetY();
    }
    public class Point implements/*$:*/ IPoint {
        public int X;
        public int GetX() { return this.X; }
        public int Y;
        public int GetY() { return this.Y; }
    }
    //$goals 2
    //$benchmark
    public void PointsNotEqual(IPoint p, IPoint q) {
        if (p != null && q != null &&
            (p.GetX() != q.GetX() || p.GetY() != q.GetY()))
            { /*$goal 0 */}
        else
            { /*$goal 1*/}
    }
\end{verbatim}

All overriden methods must be explicitly specified by \co{@Override} or \co{/*\$*/override}.
Example 4:
\begin{verbatim}
    public abstract class PointBase {
        public abstract int GetX();
        public abstract int GetY();
    }
    public class Point extends/*$:*/ PointBase {
        public int X;
        public @Override int GetX() { return this.X; }
        public int Y;
        public @Override int GetY() { return this.Y; }
    }
    //$goals 2
    //$benchmark
    public void PointsNotEqual(PointBase p, PointBase q) {
        if (p != null && q != null &&
            (p.GetX() != q.GetX() || p.GetY() != q.GetY()))
            { /*$goal 0 */}
        else
            { /*$goal 1*/}
    }
\end{verbatim}

All virtual methods must be explicitly specified by \co{@Virtual} or \co{/*\$*/virtual}.
Casts are not allowed, but can be simulated via virtual methods.

Example 5:
\begin{verbatim}
    public class Point {
        public int X;
        public @Virtual int GetX() { return this.X; }
        public int Y;
        public @Virtual int GetY() { return this.Y; }
        public @Virtual IsPlus1() { return false; }
    }
    public class PointPlus1 extends/*$:*/ Point {
        public @Override int GetX() { return this.X+1; }
        public @Override int GetY() { return this.Y+1; }
        public @Override IsPlus1() { return true; }
    }
    //$goals 2
    //$benchmark
    public void FunkyPoints(Point p, Point q) {
        if (p != null && q != null &&
            p.IsPlus1() && !q.IsPlus1() &&
            p.GetX() + q.GetX() == 42 &&
            p.GetX() - q.GetY() == 21)
            { /*$goal 0 */}
        else
            { /*$goal 1*/}
    }
\end{verbatim}

\subsection{Inline integer arrays without exceptions}

Arrays may not be parameters to a benchmark,
but they may be created as part of the benchmark code.
Jagged arrays, i.e.\ arrays of arrays, are not allowed.
Only arrays of \co{int} are allowed.

Supported operations are: Array creation, reading an array elements, writing an array element.
All operations must be guarded in the program to avoid exceptions.

\begin{verbatim}
    //$goals 2
    //$benchmark
    public void ArrayReadWrite(int size, int i, int j, int value) {
        if (size>=0 && i>=0 && i<size && j>=0 && j<size && value>0)
        {
            int[] a = new int[size];
            a[i] = value;
            if (a[j] == value)
                { /*$goal 0 */}
            else
                { /*$goal 1*/}
        }
    }
\end{verbatim}

\begin{verbatim}
    //$goals 2
    //$benchmark
    public void ArrayReadWrite(int size, int i) {
        if (size>=0 && i>=0 && i<size)
        {
            int[] a = new int[size];
            for (int j=0; j<size; j++)
                a[j] = j;
            if (a[i] == 42)
                { /*$goal 0 */}
            else
                { /*$goal 1*/}
        }
    }
\end{verbatim}

Note that we do not allow an operation that retrieves the length of an array;
as all arrays are created in the code, the length must be known to the benchmark already.