\section{Introduction}

Benchmarks have been important in the development of many areas in computer science.
Examples include databases, decision procedures (SAT, SMT, etc.), 
programming languages \cite{blackburn06dacapo}, memory management, computer architecture,
performance optimizations, etc.

At the heart of many interesting software engineering problems is the question of
reachability, i.e., whether a statement in the code may be executed in some scenario.
For example, compilers try to identify and eliminate dead code,
model checkers explore reachable model states exhaustively up to a given bound,
program verification systems (e.g., ESC/Java or Spec\# \cite{flanagan02extended,barnett04spec}) prove that faulty statements are unreachable,
and test generation tools try to trigger runtime exceptions and assertion violations
and try to achieve high code coverage.

Research in automatic software engineering tools and techniques has made a lot 
of progress in the last decades. For example, many papers have described new tools
and techniques for automatic bug finding. 
Some recent examples include JCrasher, Eclat, and Symclat for Java 
\cite{csallner04jcrasher,pacheco05eclat,damorim06empirical}
and Randoop and Pex for C\# 
\cite{pacheco07feedback,tillmann08pex}. 
What many of these papers have in common is that it is 
hard to compare them with each other. Tool implementations
target different programming languages (i.e., Java and C\#) and use different
subject programs in their evaluation. 
Lacking is a common benchmark that
can be easily used and reused by many researchers to evaluate and compare tools and techniques.

The software-artifact infrastructure repository 
(SIR)~\cite{do05supporting} is a related effort.

An existing benchmark suite can be complemented by a random benchmark generator, similar to the ones proposed for evaluating
just-in-time (JIT) compilers~\cite{goldreich86how,yoshikawa03random}.

This paper defines a common notation for writing benchmarks for tools 
that decide Reachability in Object Oriented ProgramS (\co{Roops}).
\co{Roops} is general enough to be useful for many different researchers,
working on different tools for different applications and different languages.
We currently focus on bug finding tools and automatic test case generators
for Java and C\#. But we are confident that \co{Roops} is general enough to 
support more applications.
A common benchmark notation with standardized reachability goals should allow 
the community to share and build upon each others benchmarks, and enable
researchers to compare empirical results.

The \co{Roops} notation is flexible and easily adopted for different target 
environments. The notation is a bit clumsy---but it allows targeting Java 
and C\# and possible other languages at the same time, and also allows supporting different testing 
environments such as JUnit or others. A design principle of the \co{Roops} 
notation is that such translation can be achieved by simply using regular 
expressions, not needing a full parser and type checker. In fact, we
already have translators for translating the \co{Roops} benchmarks to C\#
and to Java.

\co{Roops} includes a mechanism to identify reachability goals. 
Each \co{Roops} benchmark is simply a public method that takes parameters,
in effect being a Parameterized Unit Test\cite{} or JUnit Theory\cite{}.

\co{Roops} includes a way to describe test cases, 
so that achieved goals can be independently verified;
such test cases are Unit Tests\cite{}.

We envision that (open source) libraries will read and write programs in the
\co{Roops} notation to help newcomers build new tools.


\subsection{Competition}

The common Roops notation enables competitions,
much like other research communities do, such as
the verification community with the Satisfiability Modulo Theories Competition
(SMT-COMP)~\cite{barrett05smt-comp,barrett05design},
the cryptography community with the AES competition
\cite{nechvatal01report,daemen02design} and the programming language community with the PoplMark challenge
\cite{aydemir05mechanized}.


\subsection{Availability}

We maintain all \co{Roops} benchmarks, associated tools (benchmark generators, 
converters, etc.), and documentation under an open-source (MIT) license in the
following repository.

\verb|http://code.google.com/p/roops/|

We invite everyone to use and improve the Roops benchmarks.